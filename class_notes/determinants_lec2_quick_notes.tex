\documentclass{article}
\usepackage[utf8]{inputenc}
\usepackage[english]{babel}
\usepackage{amsmath,amsfonts,amssymb,amsthm}
\usepackage{mathtools}
\usepackage{fancyhdr}
\usepackage{commath}
\usepackage[sc,osf]{mathpazo}
\usepackage{graphicx}
\usepackage{rotating}
\usepackage{float}
\usepackage{subcaption}
\restylefloat{table}
\usepackage{multicol}
\usepackage[dvipsnames]{xcolor}
\usepackage[colorinlistoftodos]{todonotes}
\usepackage{vmargin}  % Administrar márgenes
\setpapersize{A4} % Definir tamaño del papel
\setmargins{2.5cm} % Margen izquierdo
{1cm} % Margen superior
{16.5cm} % Área de impresión horizontal
{23.42cm} % Area de impresión vertical
{15mm} % Encabezado
{5mm} % Espacio entre el encabezado y el texto
{10pt} % Pie de página
{3mm} % Espacio entre el pie de página y el texto

\pagestyle{fancy}
\fancyhf{}
\rhead{
% \includegraphics[width=4cm,height=1cm]{cropped-iitpal-at-prutor-logo.png}
}
\lhead{Determinants | Class XII}
\rfoot{}
\begin{document}
\section{Properties of Determinants}
\subsection*{Property 1}
If a determinant has all the elements zero in any row (or column) then its values is zero.
\begin{equation*}
    \Delta=
    \begin{vmatrix}
        0 & 0 & 0 \\
        a_{21} & a_{22} & a_{23} \\
        a_{31} & a_{32} & a_{33} 
    \end{vmatrix}
    =0
\end{equation*}

\subsection*{Property 2}
Determinant of a diagonal matrix is given by product of its diag entries.
\begin{equation*}
    \Delta=
    \begin{vmatrix}
        a_{11} & 0 & 0 \\
        0 & a_{22} & 0 \\
        0 & 0 & a_{33} 
    \end{vmatrix}
    =a_{11}a_{22}a_{33}
\end{equation*}
\subsection*{Property 3}
Determinant of a upper or lower triagonal matrix is given by product of its diag entries.
\begin{equation*}
    \Delta=
    \begin{vmatrix}
        a_{11} & a_{12} & a_{13} \\
        0 & a_{22} & a_{23} \\
        0 & 0 & a_{33} 
    \end{vmatrix}
    =a_{11}a_{22}a_{33}
\end{equation*}
\subsection*{Property 4}
The value of a determinant remains unaltered; if the
rows and columns are interchanged. Basically transpose of a matrix has same determinant.
\begin{equation*}
    \Delta=
    \begin{vmatrix}
        a_{11} & a_{12} & a_{13} \\
        a_{21} & a_{22} & a_{23} \\
        a_{31} & a_{32} & a_{33} 
    \end{vmatrix}
    =
    \begin{vmatrix}
        a_{11} & a_{21} & a_{31} \\
        a_{12} & a_{22} & a_{32} \\
        a_{13} & a_{23} & a_{33} 
    \end{vmatrix}
\end{equation*}
\subsection*{Property 5}
If any two adjacent rows (or columns) of a
determinant be interchanged, the value of
determinant is changed in sign only.
\begin{equation*}
    \Delta=
    \begin{vmatrix}
        a_{11} & a_{12} & a_{13} \\
        a_{21} & a_{22} & a_{23} \\
        a_{31} & a_{32} & a_{33} 
    \end{vmatrix}
\end{equation*}
\begin{equation*}
    \Delta^{'}=
    \begin{vmatrix}
        a_{21} & a_{22} & a_{23} \\
        a_{11} & a_{12} & a_{13} \\
        a_{31} & a_{32} & a_{33} 
    \end{vmatrix}
\end{equation*}
Then we have $\Delta=-\Delta^{'}$.

\subsection*{Property 6}
If a determinant has any two rows (or columns) identical or proportional, then its values is zero.
\begin{equation*}
    \Delta=
    \begin{vmatrix}
        a_{11} & a_{12} & a_{13} \\
        a_{11} & a_{12} & a_{13} \\
        a_{31} & a_{32} & a_{33} 
    \end{vmatrix}
    =0
\end{equation*}
\subsection*{Property 7}
If all the elements of any row (or column) be
multiplied by the same number, then the determinant
is multiplied by that number.
\begin{equation*}
    \Delta=
    \begin{vmatrix}
        a_{11} & a_{12} & a_{13} \\
        a_{21} & a_{22} & a_{23} \\
        a_{31} & a_{32} & a_{33} 
    \end{vmatrix}
\end{equation*}
\begin{equation*}
    \Delta^{'}=
    \begin{vmatrix}
        Ka_{11} & Ka_{12} & Ka_{13} \\
        a_{21} & a_{22} & a_{23} \\
        a_{31} & a_{23} & a_{33} 
    \end{vmatrix}
\end{equation*}
Then we have $\Delta^{'}=K\Delta$.
\subsection*{Property 8}
If each element of any row (or column) can be
expressed as a sum of two terms then the
determinant can be expressed as the sum of two
determinants, that means
\begin{equation*}
    \Delta=
    \begin{vmatrix}
        x+a_{11} & y+a_{12} & z+a_{13} \\
        a_{21} & a_{22} & a_{23} \\
        a_{31} & a_{32} & a_{33} 
    \end{vmatrix}
    =
    \begin{vmatrix}
        x & y & z \\
        a_{21} & a_{22} & a_{23} \\
        a_{31} & a_{32} & a_{33} 
    \end{vmatrix}
    =
    \begin{vmatrix}
        a_{11} & a_{12} & a_{13} \\
        a_{21} & a_{22} & a_{23} \\
        a_{31} & a_{32} & a_{33} 
    \end{vmatrix}
\end{equation*}
\subsection*{Property 9}
The value of determinant is not altered by adding to the elements of any row (or column) a constant multiple of the corresponding elements of any other row (or column).\\
\textbf{Exa: }\\
$R_1 \rightarrow R_1 + mR_2$ (change R1 as sum of R1 and m (R2).\\
$R_3 \rightarrow R_3 + nR_2$ (change R3 as sum of R3 and n (R2).

\begin{equation*}
    \Delta=
    \begin{vmatrix}
        a_{11}+ma_{21} & a_{12}+ma_{22} & a_{13}+ma_{23} \\
        a_{21} & a_{22} & a_{23} \\
        a_{31} & a_{32} & a_{33} 
    \end{vmatrix}
\end{equation*}
\begin{equation*}
    \Delta^{'}=
    \begin{vmatrix}
        a_{11} & a_{12} & a_{13} \\
        a_{21} & a_{22} & a_{23} \\
        a_{31}+na_{21} & a_{32}+na_{22} & a_{33}+na_{23} 
    \end{vmatrix}
\end{equation*}
Then this property says, $\Delta=\Delta^{'}$\\
\textbf{NOTE:} All these properties can be proved just by doing step by step algebra of determinant calculations. Take it as an exercise to prove these results for general form of 3 x 3 matrix.
\end{document}