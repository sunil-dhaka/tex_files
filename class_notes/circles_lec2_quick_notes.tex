\documentclass{article}
\usepackage[utf8]{inputenc}
\usepackage[english]{babel}
\usepackage{amsmath,amsfonts,amssymb,amsthm}
\usepackage{mathtools}
\usepackage{fancyhdr}
\usepackage{commath}
\usepackage[sc,osf]{mathpazo}
\usepackage{graphicx}
\usepackage{rotating}
\usepackage{float}
\usepackage{subcaption}
\restylefloat{table}
\usepackage{multicol}
\usepackage[dvipsnames]{xcolor}
\usepackage[colorinlistoftodos]{todonotes}
\usepackage{vmargin}  % Administrar márgenes
\setpapersize{A4} % Definir tamaño del papel
\setmargins{2.5cm} % Margen izquierdo
{1cm} % Margen superior
{16.5cm} % Área de impresión horizontal
{23.42cm} % Area de impresión vertical
{15mm} % Encabezado
{5mm} % Espacio entre el encabezado y el texto
{10pt} % Pie de página
{3mm} % Espacio entre el pie de página y el texto

\pagestyle{fancy}
\fancyhf{}
\rhead{
\includegraphics[width=4cm,height=1cm]{cropped-iitpal-at-prutor-logo.png}
}
\lhead{Circles | Class XI}
\rfoot{}
\begin{document}
\section{Position of a point wrt Circle}
To know whether a point in 2-D plane $(x_{1},y_{1})$ lies inside or outside of a circle, we can use general form of circle i.e. $x^2+y^2+2gx+2fy+c=0$:
\begin{align*}
    \text{if  } x_{1}^2+y_{1}^2+2gx_{1}+2fy_{1}+c<0 \implies \textbf{Inside}\\
    \text{if  } x_{1}^2+y_{1}^2+2gx_{1}+2fy_{1}+c>0 \implies \textbf{Outside}\\
    \text{if  } x_{1}^2+y_{1}^2+2gx_{1}+2fy_{1}+c=0 \implies \textbf{On Circle}
\end{align*}
\textbf{Note} that we just have to put value of $(x_{1},y_{1})$ and by its sign we know whether point is inside or outside or on circle.
\section{Equation of Circle}
For equation of a circle that passes through 3 non-collinear points, simply put values of points in general form of the circle. Suppose that points are $(x_{1},y_{1})$, $(x_{2},y_{2})$, and $(x_{3},y_{3})$. Since these points lie on the circle. So,
\begin{align*}
    x_{1}^2+y_{1}^2+2gx_{1}+2fy_{1}+c=0\\
    x_{2}^2+y_{2}^2+2gx_{2}+2fy_{2}+c=0\\
    x_{3}^2+y_{3}^2+2gx_{3}+2fy_{3}+c=0
\end{align*}
There are three equations and three unknowns. Simple algebra gives us values of $g,f,$ and $c$. Values are,
\begin{equation*}
    2g = \frac{[(x_1^2 – x_3^2)(y_1 – x_2) + (y_1^2 – y_3^2)(y_1 – y_2) + (x_2^2 – x_1^2)(y_1 – y_3) + (y_2^2 – y_1^2)(y_1 – y_3)]} { [(x_3 – x_1)(y_1 – y_2) – (x_2 – x_1)(y_1 – y_3)]}
\end{equation*}
\begin{equation*}
    2f = \frac{[(x_1^2 – x_3^2)(x_1 – x_2) + (y_1^2 – y_3^2 )(x_1 – x_2) + (x_2^2 – x_1^2)(x_1 – x_3) + (y_2^2 – y_1^2)(x_1 – x_3)] }{[(y_3 – y_1)(x_1 – x_2) – (y_2 – y_1)(x_1 – x_3)]}
\end{equation*}
To get $c$ simply put values of $g$ and $f$ in,
\begin{equation*}
    c=-x_1^2-y_1^2-2gx_1-2fy_1
\end{equation*}
With the help of $g,f,c$ we can get center of circle i.e. $(-g,-f)$ and radius i.e. $\sqrt(g^2+f^2-c)$. Note that one can try to remember above formulas using the pattern that is in them.
\end{document}