\documentclass{article}
\usepackage[utf8]{inputenc}
\usepackage[english]{babel}
\usepackage{amsmath,amsfonts,amssymb,amsthm}
\usepackage{mathtools}
\usepackage{fancyhdr}
\usepackage{commath}
\usepackage[sc,osf]{mathpazo}
\usepackage{graphicx}
\usepackage{rotating}
\usepackage{float}
\usepackage{subcaption}
\restylefloat{table}
\usepackage{multicol}
\usepackage[dvipsnames]{xcolor}
\usepackage[colorinlistoftodos]{todonotes}
\usepackage{vmargin}  % Administrar márgenes
\setpapersize{A4} % Definir tamaño del papel
\setmargins{3cm} % Margen izquierdo
{1cm} % Margen superior
{15cm} % Área de impresión horizontal
{23.42cm} % Area de impresión vertical
{15mm} % Encabezado
{7mm} % Espacio entre el encabezado y el texto
{10pt} % Pie de página
{3mm} % Espacio entre el pie de página y el texto

\pagestyle{fancy}
\fancyhf{}
\rhead{
\includegraphics[width=4cm,height=1cm]{cropped-iitpal-at-prutor-logo.png}
}
\lhead{Circles | Class XI | Exemplar Problems}
\rfoot{}
\begin{document}
\section*{Practice Questions}
\paragraph{Q1.}Find the centre and radius of the circle $x^2 + y^2 – 2x + 4y = 8$. 
\begin{flushright}
Page-193
\end{flushright}
\paragraph{Q2.}Fill Blank: The equation of the circle which passes through the point (4, 5) and has its centre at (2, 2) is \rule{3cm}{0.2mm}. 
\begin{flushright}
Page-200
\end{flushright}
\paragraph{Q3.}True/False: Circle on which the coordinates of any point are $(2 + 4 cos\theta, –1 +
4 sin\theta)$ where $\theta$ is parameter is given by $(x – 2)^2 + (y + 1)^2 = 16$. 
\begin{flushright}
Page-199
\end{flushright}
\paragraph{Q4.}The equation of the circle in the first quadrant touching each coordinate axis at a distance of one unit from the origin is:
\begin{enumerate}
    \item $x^2 + y^2 – 2x – 2y + 1= 0$
    \item $x^2 + y^2 – 2x – 2y – 1 = 0$
    \item $x^2+ y^2 – 2x – 2y = 0$
    \item $x^2 + y^2 – 2x + 2y – 1 = 0$
\end{enumerate}
\begin{flushright}
Page-197
\end{flushright}
\clearpage






\section*{Solution and Hints}
\paragraph{S1.}\
\textbf{Approach-1} If one knows general form of circle and its center and radius formulas then it is an easy one. Compare with $x^2+y^2+2gx+2fy+c=0$,
\begin{align*}
    center=(-g,-f)=(1,-2)
\end{align*}
Using radius formula,
\begin{equation*}
    r=\sqrt{g^2+f^2-c}=\sqrt{1^2+(-2)^2-(-8)}=\sqrt{13}
\end{equation*}
\textbf{Approach 2}
We write the given equation in the form $$(x^2 – 2x) + ( y^2 + 4y) = 8$$
Now, completing the squares, we get
\begin{align*}
    (x^2 – 2x + 1) + ( y^2 + 4y + 4) = 8 + 1 + 4
    (x – 1)^2 + (y + 2)^2 = 13
\end{align*}
Comparing it with the standard form of the equation of the circle, we see that the
centre of the circle is (1, –2) and radius is 13.

\paragraph{S2.}
 As the circle is passing through the point (4, 5) and its centre is (2,2) so its radius is
 $$ r^2=(4 - 2)^2 + (5 - 2)^2 = 13$$
 Therefore the required answer is, in center-radius form
\begin{equation*}
    (x – 2)^2 + (y – 2)^2 = 13
\end{equation*}

\paragraph{S3.}
This is a question based on parametric form and center-radius form of circles. Recall parametric form from class notes.
\begin{align*}
    x=2 + 4 cos\theta \implies (x – 2) = 4 cos\theta \\
    y = –1 + 4 sin\theta \implies y + 1 = 4 sin\theta
\end{align*}
Now square both equations and we get,
\begin{equation*}
    (x – 2)^2 + (y + 1)^2 = 16
\end{equation*}
It is a \textbf{TRUE} statement.
\paragraph{S4.} It is a simple extension of Que 1. One way to solve is to simply write every equation in center radius form and then see which one gives a circle touching both the axes with its centre (1, 1) and
radius one unit. Clearly, this points to option A.
\begin{align*}
    x^2 + y^2 – 2x – 2y + 1= 0 \\
    \implies (x-1)^2+(y-1)^2=1
\end{align*}
Another approach is to get center and radius of circles by comparing them with the general form of circle. This approach is quick because all the options are given in general form of particular circles. This gives following center and radius,
\begin{enumerate}
    \item (1,1) and 1
    \item (1,1) and $\sqrt{3}$
    \item (1,1) and $\sqrt{2}$
    \item (1,-1) and $\sqrt{3}$ 
\end{enumerate}
This approach also gives option (A).
\end{document}